\documentclass[journal,onecolumn,12pt]{IEEEtran} 

\usepackage{amsmath,amssymb,bm}
\usepackage{amsthm, amsfonts}	
\usepackage{bm,bbm}
\usepackage[normalem]{ulem}
\usepackage{color}
\usepackage{fancybox}
\usepackage{url,booktabs}
\usepackage[round]{natbib}

\usepackage{xr}



\title{Reply to Reviewer's Comments on\\
''BLABLABLABLABLA''}
\author{}

\begin{document}

\maketitle
\pagenumbering{roman}
\setcounter{page}{1}

ACA SE PONE UN AGRADECIMIENTO 

We are grateful to the reviewer for pointing out relevant issues in our manuscript.

In the following, we discuss how we dealt with each raised issue. 

\vskip+1ex
\noindent \dotfill

\section*{\fbox{Reviewer \#1 Transcript:}}

ACA SE PEGA TODO EL TEXTO VERBATIM DEL JURADO

\subsection*{\ovalbox{General Comments}}
Aca se ponen cada uno de los puntos de ese texto, o por ejemplo se agrupan los que están relacionados.
\vskip+1ex
\noindent \dotfill
\vskip+1ex
%
\begin{quotation}
{\color{blue}
A cada punto hay que poner algo que se hace como:

Modificamos la figura 6, agregamos lo que sugiere el jurado.  Además modificamos la seccion 4.3 aclarando en detalle el punto mencionado.
BLA BLA BLA

Para los puntos en los que NO están de acuerdo lo que plantea el jurado, se puede decir que se hace una aclaración, se explica mejor, se ofrecen más detalles pero SE DEFIENDE la postura.


}
\end{quotation}
%
\vskip+1ex
\noindent \dotfill
\vskip+1ex

fdsfsdf

\vskip+1ex
\noindent \dotfill
\vskip+1ex
%
\begin{quotation}
{\color{blue}
We have now included more participants to this experiment and it was divided in two modalities.  The first, which is now called passive-modality, includes now 4 subjects.  The second, an active-modality, now includes 4 more participants who were engaged in a copy-spelling task on a P300-Based BCI Speller.   For these subjects, the pseudo-real dataset is constructed by superimposed a P300 template obtained for each subject into segments marked as hit.  The baseline for each segment is constructed from a random no-hit segment for the same subject.  
}
\end{quotation}
%
\vskip+1ex
\noindent \dotfill
\vskip+1ex
fdsfsfsdf


\vskip+1ex
\noindent \dotfill
\vskip+1ex
%
\begin{quotation}
{\color{blue}
fdfdsfs
}
\end{quotation}
%
\vskip+1ex
\noindent \dotfill
\vskip+1ex

fdsfsfs
\vskip+1ex
\noindent \dotfill
\vskip+1ex
%
\begin{quotation}
{\color{blue}

Absolutely thank you very much for providing these very important references.  We added all of them.

}
\end{quotation}
%
\vskip+1ex
\noindent \dotfill
\vskip+1ex

fdsfsfds
\vskip+1ex
\noindent \dotfill
\vskip+1ex
%
\begin{quotation}
{\color{blue}

We have included a new paragraph describing the depictions which are shown in the table and in which areas are they used for.  

}
\end{quotation}
%
\vskip+1ex
\noindent \dotfill
\vskip+1ex

5. In Figure 2 please insert the y-axis title to clearly present ERP template.

\vskip+1ex
\noindent \dotfill
\vskip+1ex
%
\begin{quotation}
{\color{blue}
Our applogize for this mistake.  We have corrected it and revised all the other figures.
}
\end{quotation}
%
\vskip+1ex
\noindent \dotfill
\vskip+1ex

fdsafdafafsdf

\vskip+1ex
\noindent \dotfill
\vskip+1ex
%
\begin{quotation}
{\color{blue}
fdafafa

 }
\end{quotation}
%
\vskip+1ex
\noindent \dotfill
\vskip+1ex

dfsafa

\vskip+1ex
\noindent \dotfill
\vskip+1ex
%
\begin{quotation}
{\color{blue}
fdfafas
}
\end{quotation}
%
\vskip+1ex
\noindent \dotfill
\vskip+1ex

dfafasf
\vskip+1ex
\noindent \dotfill
\vskip+1ex
%
\begin{quotation}
{\color{blue}
We have now included more subjects and decided to remove the statistical analysis sections which was prepared for only one subject.
}
\end{quotation}
%

\bibliographystyle{mdpi}
\bibliography{article}

\end{document}